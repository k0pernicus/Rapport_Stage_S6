\documentclass{report}

\usepackage[utf8]{inputenc}
\usepackage[T1]{fontenc}
\usepackage[francais]{babel}
\usepackage{hyperref}
\usepackage{color}
\usepackage{calc}
\usepackage{pseudocode}
\AddThinSpaceBeforeFootnotes
\FrenchFootnotes

\title{Rapport de stage}
\author{Antonin Carette}
\date{Compilé le \today}

\begin{document}

\maketitle

\tableofcontents


\chapter{Introduction}

\section{Remerciements}
Je remercie tout d'abord l'équipe Bonsai, pour toute l'accueil généreuse recueillie durant ces 3 mois de stage.
\newline 
Je remercie ensuite les membres du projet \textit{Vidjil}: \textbf{Giraud Mathieu}, \textbf{Salson Mikaël} ainsi que \textbf{Duez Marc} pour l'intégration mais aussi l'aide avec laquelle chaque membre m'a aidé pour mon sujet de stage.
\newline
Aussi, je remercie M. \textbf{Meftali Samy} ainsi que M. \textbf{Giraud Mathieu} pour toute l'attention donnée quant au travail sur le projet, ainsi que sur ce rapport.
\newline
Pour finir, je tiens à remercier ma famille pour tout le soutien apporté durant ces trois mois de stage, ainsi qu'à mes collèges de bureau: \textbf{Dufresne Yoann} et \textbf{Vrolland Christophe}.

\section{Présentation de l'équipe Bonsai}
Bonsai a été créé en 2011 par Mme \textbf{Touzet Hélène}, responsable de l'équipe; elle était, auparavant, nommée Sequoia.
\newline
Cette jeune équipe dépend du \textit{L.I.F.L.}, de l'\textit{I.N.R.I.A.} ainsi que du \textit{C.N.R.S.}.
\newline
Mes encadrants ont été, dans l'équipe, M. \textbf{Giraud Mathieu}, M. \textbf{Salson Mikaël} ainsi que M. \textbf{Duez Marc}.
\newline
Les travaux des membres de l'équipe (on en compte actuellement 22) sont assez larges, tous réunis dans un seul domaine: la biologie.
\newline
En effet, l'équipe Bonsai est très réputée dans la bio-informatique; on dénombre plus de 30 applications/logiciels utilisables gratuitement, créés par les membres de l'équipe.
\newline
Cette équipe m'a donc accueillie pour mon stage de fin de licence, se déroulant du 1er Avril au 30 Juin inclus.


\chapter{Contexte du stage et objectifs}

\section{La Leucémie Aigüe Limphoblastique}
La Leucémie Aigüe Limphoblastique (\textit{LAL}) est un cancer liquide\footnote{Le cancer liquide, encore appelé cancer sanguin, est un ensemble composé des leucémies (cancers du sang et de la moëlle épinière) et des lymphomes (cancers du système lymphatique).} affectant majoritairement les enfants.
\newline
Cette leucémie peut-être induite par un défaut dans la recombinaison V(D)J, mécanisme de recombinaison de l'ADN présents chez les Humains et autres vertébrés permettant de créer une grande diversité de récepteurs d'anti-corps, nécessaires à la reconnaissance d'antigènes étrangers, pouvant apporter diverses pathologies plus ou moins graves.
\newline
\textcolor{red}{-> Tous les outils permettant d'évaluer les recombinaisons V(D)J; explication des clônes}

\section{Le projet Vidjil}

\textit{Vidjil} est un projet intra (pour le développement des programmes) et extra (pour toute la partie échantillonage) Universitaire créé par un petit groupe de chercheurs et d'ingénieurs de l'équipe Bonsai, mais aussi du Laboratoire d'Hématologie de Lille2.
\subsection{Contexte}
Le projet \textit{Vidjil} a pris naissance lors d'une problèmatique assez simple: peu d'outils fiables et complets étaient disponible pour le séquençage à haut-débit, demandé par l'équipe d'Hématologie de Lille2 pour ses analyses concernant principalement la \textit{LAL}, menée par M. \textbf{Preudhomme Claude} - requièrant beaucoup de travail, au niveau algorithmique et d'adaptation à un modèle biologique\footnote{Voir le premier article écrit par M. \textbf{Giraud Mathieu} et M. \textbf{Salson Mikaël}, sur l'\textit{algorithme}.}.
\newline
Après en avoir parlé avec M. \textbf{Figeac Martin}, il a de suite émis l'idée d'en discuter avec des informaticiens intéressés par la biologie, issus de l'équipe Bonsai du L.I.F.L. - ce qu'il a fait en discutant de tout celà avec M. \textbf{Giraud} et M. \textbf{Salson}.
\newline
Les deux équipes se sont donc rencontrés début de l'année 2011 pour en discuter calmement, et essayer de mieux décortiquer la problèmatique principale. Ce n'est qu'en 2012 que le premier programme a été écrit, et que le projet \textit{Vidjil} a réellement débuté.

\subsection{Principe}
Ce projet consiste à la mise en place d'un programme en langage C++, appelé \textit{algorithme}, ainsi que d'une interface Web, appelé \textit{afficheur}, permettant de calculer et afficher plusieurs informations quand aux recombinaisons V(D)J de patients souffrants de la maladie précédemment évoquée, à travers un jeu de données précis.
\newline
\textcolor{red}{-> Expliquer en bas de page comment ont été réalisés ces jeux de données + comment les exploiter}
\newline
Toutes ces informations pourront être, par la suite, utilisées par les services hôpitaux/biologiques afin de traiter les données recueillies directement sur les patients, les étudier, et pouvoir prédire les rechutes quant à cette maladie.

\subsection{L'équipe}
\textit{Vidjil} est un projet créé en partenariat avec le \textit{Laboratoire d'Hématologie de Lille2}, par M. \textbf{Giraud Mathieu}, M. \textbf{Salson Mikaël} et M. \textbf{Preudhomme Claude}, responsable de l'équipe d'Hématologie de Lille2.
\newline
L'équipe de ce projet travaille désormais avec \textit{EuroClonality NGS} (un Consentium Européen contenant plusieurs laboratoires, ayant un grand intérêt pour le projet quant à l'échantillonage), ainsi que différents laboratoires à l'intérieur mais aussi à l'extérieur de la France comme celui de Paris, de Rennes, mais aussi de République Tchèque et d'Angleterre.

\section{Objectifs de stage}
Mes objectifs ont été clairs, et m'ont été donné par M. \textbf{Giraud Mathieu} et M. \textbf{Salson Mikaël} lors d'un entretien téléphonique et oral en Décembre 2013/Janvier 2014.

\subsection{L'\textit{afficheur}}
Pour un premier, je devais intégrer dans l'\textit{afficheur} un graphe permettant de visualiser les distances d'édition des clônes, les uns par rapport aux autres.
\newline
Ce premier travail devait s'effectuer sur une durée de deux mois, avec une programmation en HTML5, CSS3 et Less, le langage orienté objet Javascript, Ajax, et en utilisation les frameworks Javascript D3JS et JQuery, langages déjà intégrés au projet depuis sa conception par \textbf{Marc Duez}.
\subsection{L'algorithmique (DBSCAN)}
Par la suite, le travail devait déborder sur une partie beaucoup plus algorithmique avec l'utilisation de DBSCAN\footnote{DBSCAN est un algorithme de partitionnement de données proposé en 1996 par \textbf{Martin Ester}, \textbf{Hans-Peter Kriegel}, \textbf{Jörg Sander} et \textbf{Xiaowei Xu}}, jusqu'à la fin de mon stage.
\newline
\textcolor{red}{-> A VOIR ET DÉVELOPPER!!!}


\chapter{L'\textit{afficheur}}

\section{Un travail d'un an...}
Toute la conception et l'écriture de l'\textit{afficheur} a été faite par l'ingénieur de recherche du projet \textit{Vidjil} \textbf{Marc Duez}.
\newline
M. \textbf{Duez} a commencé à participer au projet dès Mars/Avril 2013, projet de fin d'année pour son Master 2.
\newline
Son ambition était de créer un logiciel répondant aux besoins des utilisateurs (principalement des biologistes), avec le plus de facilité possible, tout en respectant les principes de "Programmation Orientée Objet" et de "Modèle-Vue-Contrôleur".

\section{Langages et frameworks}
À mon arrivée dans l'équipe, le programme utilisait les langages les plus simples et les plus utilisés du Web: HTML5, CSS3, Javascript et Ajax.
\newline
Tout le côté "Administration serveur" a été réalisé avec le langage Python (version 2.7) grâce au framework open source web2py.
\newline
Côté conception avancée, il y avait plus de 20 classes Javascript déjà écrites (à environ 1000 lignes de code par classe), utilisant 2 frameworks Javascript reconnus (D3JS et JQuery), ainsi que le langage Less\footnote{Less est un langage dynamique de génération de feuilles de style, conçu par \textbf{Alexis Sellier}.} - langage très utile quant au changement dynamique de feuille de style afin de mieux visualiser les données calculées pour un échantillonage préparé, et permettant beaucoup plus de souplesse que le langage CSS.

\section{Composition de l'\textit{afficheur}}

\textcolor{red}{-> Screenshots de l'afficheur et affichage des différentes parties; quelle partie appartient à quel fichier, pourquoi, à quoi sert cette partie?}

\section{Préparation}
Tout ceci s'est faite durant la 1ère semaine de stage, et la moitié du temps durant les deux suivantes.

\subsection{La documentation}
Ne connaissant pas D3JS, JQuery, Ajax ou encore Less, il a donc fallu que je me documente à leur sujet, exploiter toutes les fonctionnalités mais aussi comment les intégrer, et quel est le but recherché quant à l'utilisation de tel framework ou tel langage.

\subsection{L'intégration au projet}
Je ne pouvais me permettre d'ajouter directement ma nouvelle fonctionnalité dans la classe correspondant à l'endroit où elle devait apparaître dans l'\textit{afficheur}, il m'a donc fallu tout d'abord étudier le code ainsi que recourir à la création de la documentation pour le développeur non-écrite ou obsolète, ainsi que corriger quelques bugs mineurs dans l'\textit{afficheur} afin de mieux pouvoir discerner quelle partie du projet était intégrée dans quelle classe objet.
\newline
De plus, il était très important pour moi de respecter les règles de l'équipe en matière d'ingénièrie logicielle, ne serait-ce par l'écriture du code, de la documentation, mais aussi par rapport au respect des commits/push sur le serveur, via l'utilitaire \textit{Git}.

\subsection{Le domaine biologique}
Dernière chose, je me suis remis - lors des deux premières semaines - à la biologie et l'étude de la recombinaison V(D)J chez le vertébré, mais aussi les techniques de séquençage et de clusterisation afin de mieux comprendre et poursuivre les discussions autour du sujet, lors des réunions dans l'équipe, ou avec le Laboratoire d'Hématologie de Lille2.

\section{Le graphe}
La distribution quant à l'édition de distance entre les différents clônes est très intéressante afin de visualiser les rapprochements entre ceux-ci, et permettre une visualisation par échelle de couleur, afin de mieux distancer quels sont les clônes les plus similaires par rapport aux autres.
\newline
Le projet contenait à lui seul d'ores-et-déjà 5 distributions:
\begin{itemize}
\item V/J génique
\item V/J allèlique
\item V/J selon la distance des clônes
\item la visualisation l'abondance des clônes V et J
\item la visualisation des V selon un histogramme
\end{itemize}
Toutes ces distributions sont utiles et nécessaires à la visualisation des clônes dans un environnement précis, en fonction d'un jeu de données.

\subsection{Ajout d'une nouvelle distribution}
L'ajout d'une nouvelle distribution n'était pas aisée - en effet, il m'a fallu me mettre dans le code afin de prévoir où implanter ma distribution, et quelles modifications effectuer sur la partie de l'\textit{afficheur} utilisé: le \textit{scatterPlot}.
\newline
De plus, par rapport aux autres distributions effectuées par l'ingénieur de recherche de la petite équipe, il fallait forcer un peu plus le moteur physique et 3D de D3JS afin que lui seul puisse positionner les clônes le plus précisémment possible par rapport à une distance, déjà calculée et implantée sous la forme d'un tableau JSON - les points prendront donc une position naturelle, on ne la calculera pas et ne la donnera pas au moteur afin qu'il puisse les placer.
\newline
Autre soucis à prévoir, la non-possibilité pour D3JS de calculer absolument tous les déplacement d'arêtes possibles dans le graphe - en effet, ayant un jeu de données calculé pour 100 clônes, celà nous fera donc 10000 arêtes possibles pour l'ensemble du graphe - la possibilité de calculer par frame la position de chacune d'elle est compliquée, l'optimisation du moteur n'ayant pas été étudié.

\subsection{Création du graphe}
D3JS contient, pour son moteur, absolument toutes les fonctions nécessaires afin de construire proprement un graphe - le soucis concernera plus l'exactitude des informations données.
\newline
\textcolor{red}{-> Théorie des graphes et exactitude de la représentation}
\newline
\textcolor{red}{-> Screenshots de l'établissement du graphe, pas-à-pas}

\subsection{Problème et résolution des arêtes}
La technologie D3JS est utilisée à fort escient dans le projet, d'ores-et-déjà; il était donc tout à fait naturel de le continuer en utilisant celle-ci pour la création et l'exploitation du graphe à mettre en oeuvre.
\newline
Cependant, une question restait encore à éclaircir: sur un graphe à 100 clônes, chacun était lié aux 99 autres par une arête, il faudra donc créer un graphe à 10000 arêtes; la question est: "Le moteur du framework est-il capable de supporter absolument tout le calcul nécessaire pour la mouvance du graphe?"
\newline
La réponse est \textbf{non}, le framework n'en est pas capable, car il n'a pas été créé/optimisé pour cette tâche - nous allons voir par la suite comment il a fallu optimiser un maximum la composition du graphe afin de le représenter plus rapidement, avec un maximum de concordance.

\subsubsection{Il était une fois, un premier algorithme...}
Le premier algorithme qui m'est venu à l'esprit est de supprimer toutes les arêtes ayant une longueur supérieure à une distance donnée, directement dans le programme - cela permettra donc d'alléger le tableau d'arêtes prit en compte par le moteur, et de moins le faire souffrir lors des calculs.
\newline
Ce procédé a été un semi-échec: nous avons supprimé un peu plus de 50\% des arêtes disponibles (nous étions passé de 10000 à 4970 sur un jeu de données) - malheureusement, le moteur peinait à calculer toutes les positions des noeuds et arêtes par frame une nouvelle fois (sur une machine à processus corei5 à 8Go de RAM, GPU Intel HD 4000).
\newline
Il était donc nécessaire de chercher un nouvel algorithme qui pourrai, au moins, diviser de moitié le nombre d'arêtes enregistrées/calculées par le moteur.

\subsubsection{L'algorithme contre-attaque}
Le deuxième algorithme a été trouvé par \textbf{Marc Duez}; il s'agit de:
\begin{enumerate}
\item Trouver les 3 plus grandes arêtes du graphe ayant les caractèristiques suivantes:
	\begin{itemize}
	\item chaque arête ne doit pas avoir de point commun avec les autres
	\item chaque point à une distance minimale d'une extrêmité d'une arête trouvée ne doit plus exister, lors du calcul
	\item chaque arête doit être le moins parallèle possible des autres
	\end{itemize}
\item calculer toutes les arêtes possibles entre une extrémité d'arête trouvée
\item calculer 4/5 arêtes possibles, dans une distance minimale, des points trouvés lors de l'étape 2
\end{enumerate}
Pour le résultat, il a été concluant - en effet, en suivant à la main l'algorithme, nous pouvons voir que:
\begin{itemize}
\item l'étape 1 se fera avec un maximum de 3 arêtes
\item l'étape 2 se fera avec un maximum de 600 arêtes (au total: 600 + 3 arêtes)
\item l'étape 3 se fera avec un maximum de 100 * 5 arêtes (au total: 1103 arêtes)
\end{itemize}
Ainsi, nous avons pu enlever un peu moins de 90\% des arêtes totales, en moyenne; ce qui est très largement au-delà de ce qu'il nous faut pour le temps de calcul, mais aussi pour l'allégement du moteur de D3JS.
\newline
Malheureusement, nous étions confronté à un problème, non pas par rapport au temps, mais par rapport à la fidélité de la représentation du graphe.
\newline
En effet, notre représentation préserve bien, grâce au paramètrage de l'élasticité donnée à chaque arête dans le moteur D3JS, les distances les plus faibles (inférieures à un certain pourcentage de comparaison) - le soucis étant maintenant que les plus grandes distances ne sont pas respectées. Ainsi, deux points très éloignés, à cause du paramètrage de l'élasticité, peuvent maintenant être très proches dans le graphe.
\newline
Afin de palier à celui-ci, il a fallu garder la vision des arêtes pour un/plusieurs point(s) sélectionné(s), afin de bien visualiser qui est proche, et qui est loin d'un autre point. Chaque arête sera visualisé en fonction des points présents, mais aussi en fonction de leur longueur - si l'arête détient une longueur supérieure à un pourcentage bien défini par avance, il ne pourra être affiché, ce qui permettra de ne pas afficher un graphe complet pour chaque point, ce qui rendrai la vision du graphe très complexe, mais aussi d'afficher surtout les points distancés d'une longueur voulue/intéressante.

\textcolor{red}{-> Résolution de cette approximation en modifiant les propriétés du moteur D3JS + élimination des clônes autour d'un des 3 grandes arêtes au dessus d'une longueur de 'n'}
\newline

\chapter{DBSCAN}

Ce chapitre conduira mon travail ainsi que mes réflexions sur la deuxième partie du stage que je devais effectué, sur l'intégration de l'algorithme de clusterisation DBSCAN (\textit{Density-Based Spatial Clustering of Applications with Noise}).

\section{Présentation générale}

\subsection{Présentation}
DBSCAN est un algorithme permettant la clusterisation de données, proposé en 1996 par Martin Ester, Hans-Peter Kriegel, Jörg Sander et Xiaowei Xu, s'appuyant sur la densité estimée des clusters à partitionner.
\newline
Cet algorithme est très simple à implémenter.
\newline
Tout d'abord, il faut choisir deux paramètres essentiels:
\begin{itemize}
\item une distance ($\epsilon$), représentant un rayon de cercle, qui nous aidera à considérer que les points situés à une distance inférieure ou égale à celle-ci d'un autre est représentatif d'un seul cluster
\item un nombre minimum de points (\textit{Mp}), que l'on utilisera par la suite pour considérer si le point prit en compte est considéré comme un \textbf{point central} (\textit{\textbf{core point}} - le point détient un nombre de voisins plus élevé que \textit{Mp}, dans le rayon $\epsilon$), un \textbf{point de bordure} (\textit{\textbf{border point}} - le point détient un nombre de voisins moindre que \textit{Mp}, mais est voisin d'un \textbf{point central}), ou encore un \textbf{point de bruit} (\textit{\textbf{noise point}} - tout point n'étant pas l'un des deux précédemment évoqués).
\end{itemize}
Cet algorithme est ici intéressant car il nous permet donc bien de parcourir chacun des points d'un graphe, son voisinage, et de clusteriser chaque point en fonction de son voisinage, et des deux paramètres énoncés précédemment - la densité revient donc au nombre de points (\textit{Mp}) compris dans un rayon bien spécifié ($\epsilon$). Tout ceci va nous permettre de pouvoir modifier l'\textit{afficheur} afin de pouvoir maximiser le nombre d'outils utiles aux chercheurs - notamment via la colorisation des clusters, ou encore l'apport des regroupements de clones dans le graphe.
\newline
Il est intéressant ici de constater ici que DBSCAN clusterise les données selon une certaine densité, donnée en paramètre via epsilon et \textit{Mp}; cet algorithme ne pourra donc pas gérer les clusters de densités différentes (ce qui ne nous intéresse pas ici, fort-heureusement).

\subsection{L'algorithme}

L'algorithme est représenté ci-dessous (issu de Wikipedia\footnote{\url{http://fr.wikipedia.org/wiki/DBSCAN}}):

\begin{pseudocode}{DBSCAN}{D, eps, MinPts}
\COMMENT{Base de l'algorithme}\\
C $=$ 0\\
$Pour chaque point $P$ non visité des données $D\\
$marquer $P$ comme visité$\\
PtsVoisins$ = $\CALL{epsilonVoisinage}{P, eps}\\
\IF tailleDe(PtsVoisins) < MinPts
\THEN $marquer $P$ comme "bruit"$
\ELSE 
	\BEGIN
		C$++$\\
		\CALL{extensionCluster}{D, P, PtsVoisins, C, eps, MinPts}\\
	\END\\
\\

\COMMENT{Procédure permettant d'inclure un point dans un cluster - extension d'un cluster}\\
\PROCEDURE {extensionCluster}{D, P, PtsVoisins, C, eps, MinPts}
	$Ajouter $P$ au cluster C$\\
	\FOREACH P' \in PtsVoisins \DO
			\IF P'$ n'a pas été visité$
			\THEN 
				\BEGIN
					$Marquer $P'$ comme visité$\\
					PtsVoisins'$ = $ \CALL{epsilonVoisinage}{D, P', eps}\\
					\IF tailleDe(PtsVoisins') >= MinPts
					\THEN PtsVoisins$ = $PtsVoisins \cup PtsVoisins'\\
				\END\\
			\IF P'$ n'est membre d'aucun cluster$
			\THEN $Ajouter $P'$ au cluster $C
\ENDPROCEDURE

\COMMENT{Procedure retournant tous les points de $D$ qui sont à une distance inférieure à $eps$ de $P$}\\
\PROCEDURE{epsilonVoisinage}{D, P, eps}
	$Retourner tous les points de $D$ qui sont à une distance inférieure à $eps$ de $P
\ENDPROCEDURE
\end{pseudocode}

\section{Implantation de l'algorithme}

\subsection{Choix quant aux paramètres}

Les paramètres pris par défaut sont de 0 pour $\epsilon$, ainsi que pour \textit{Mp}. Ainsi, au départ, chaque clone a son propre cluster.
\newline
Le meilleur moyen pour faire varier les paramètres et réaliser, pour soi, la meilleure clusterisation, est de faire varier les paramètres par nos propres moyens.
\newline
J'ai donc ajouté, pour cela, deux sliders dans l'\textit{interface}:
	\begin{itemize}
	\item{Un slider pour faire varier la distance prise en compte entre chaque clone: $\epsilon$},
	\item{Un slider pour faire varier le nombre de voisins minimum pris en compte dans l'algorithme: \textit{Mp}.}
	\end{itemize}
Ces sliders calculent, à chaque modification apportée par clic, un nouvel objet DBSCAN selon les nouveaux paramètres demandés - ce qui permet de visualiser en temps réel les modifications apportées manuellement.
\newline
La visualisation des clusters se fait par couleur - en effet, chaque clone appartient à un cluster bien précis, calculé et attribué lors du calcul de l'algorithme, chaque cluster ayant une couleur bien spécifique, attribuée du début à la fin de la visualisation.

\subsection{Visualisation}

La visualisation des groupes de clones est primordiale - elle en est même la base du sujet de stage.
\newline
Il a donc fallu travailler sur la visualisation de chaque clone d'un cluster, indépendamment des autres, mais aussi de l'intéraction d'un clone avec un autre, présent ou non dans le même cluster.
\newline
D'où l'idée d'implémenter trois choses, accomplissant parfaitement la visualisation et l'intéraction avec les clusters:
	\begin{itemize}
	\item{Une nouvelle colorisation, afin de visualiser rapidement quel clone est présent dans le même cluster qu'un autre,}
	\item{Une nouvelle clusterisation, afin de visualiser rapidement, via le menu gauche, quels sont les différents groupes de clones existant,}
	\item{Une nouvelle distribution, afin de pouvoir interagir physiquement, via un nouveau paramètrage du moteur physique D3JS, avec un ou plusieurs clones ou cluster - selon la clusterisation choisie pour l'étude.}
	\end{itemize}

\subsubsection{Ajout d'une nouvelle colorisation}

La colorisation s'est imposée d'elle-même, étant beaucoup plus bénéfique visuellement et résiduelle dans le temps (il est en effet possible de colorier les clones selon les résultats de DBSCAN, tout en étudiant une distribution précise, autre que celle de DBSCAN).
\newline
Aussi, il a fallu faire attention à plusieurs choses:
	\begin{itemize}
		\item{La couleur choisie pour chaque cluster est comprise entre 0 et 270 - afin d'aller d'une teinte rouge (0) vers le violet (270), des couleurs déjà utilisées dans le projet, et permettant aussi de respecter scrupuleusement la légende des couleurs déjà établie par l'ingénieur de recherche du projet}
		\item{Le dégradé est de mise - il a fallu faire attention à ce que les noeuds se ressemblant n'aient pas une teinte beaucoup trop proche l'une de l'autre.
			\newline
			Pour palier à ce problème, j'ai donc créé un objet tableau à N entrées (100 ici, pour utiliser au maximum 100 clusters, soit 1 clone par cluster) que j'ai expressément mélangé afin d'obtenir un tableau de clusters avec une suite de couleurs non-dégradée, via l'algorithme de Fisher-Yates (encore appelé mélange de Knuth), utilisé dans ce genre de situation.}
	\end{itemize}

\subsubsection{Ajout d'une nouvelle clusterisation}

L'algorithme DBSCAN est un algorithme de clusterisation - il était donc normal qu'il faille ajouter, dans l'\textit{interface}, la clusterisation DBSCAN afin de pouvoir visualiser directement les clusters, indépendamment de la colorisation et de la distribution étudiée.
\newline
La classe "Model" du projet a donc été légèrement modifiée pour ceci, afin de permettre non seulement la visualisation générale de cette clusterisation, mais aussi la visualisation de toutes les mises-à-jour futures de celle-ci, lors des changements des paramètres $\epsilon$ et \textit{Mp} de l'algorithme par l'utilisateur.

\subsubsection{Ajout d'une nouvelle distribution}

Pourquoi a-t-il fallu créer/ajouter une nouvelle distribution, au lieu d'utiliser celle déjà établie avec le graphe représentant les distances d'édition ?
\newline
Tout d'abord, il ne s'agira pas d'un seul graphe, mais de tout un ensemble de petits graphes représentant chacun un cluster, présents dans la même fenêtre SVG - ainsi, il faut prendre en compte que les distances entre chaque point du graphe ne sera pas représenté, mais que l'on veut obtenir un graphe complet par cluster.
\newline
Ensuite, les distances calculées et délivrées par le programme en C++ ne seront pas prises en compte! Nous voulons que chaque cluster soit condensé/renfermé - ainsi, il faudra donner une valeur prise par défaut mais minimaliste, à chaque lien entre une source et une cible \textbf{appartenant au même cluster}.
\newline
Vient après une suite de légères modifications:
	\begin{itemize}
	\item{Une augmentation de la gravité,}
	\item{Une élasticité non-utilisée,}
	\item{L'ajout d'une modification de la distribution en temps réel, en fonction des paramètres utilisés par l'utilisateur, et pouvant être modifié à n'importe quel moment.}
	\end{itemize}
Tout ceci a nécessité la création d'un nouveau paramètre du moteur physique D3JS, spécialement conçu pour la plus précise visualisation intéractive des clusters.
\newline
Heureusement, certaines choses ont été gardés du moteur physique implémenté pour le graphe de distance d'édition, notamment la disposition du cadre SVG, son remplissage et l'intéraction établie avec l'ensemble de l'\textit{interface}, comme l'effacement des légendes, la modification de celles-ci manuellement, etc...

\subsection{Résultats}

Les résultats de cette implémentation de l'algorithme DBSCAN en JavaScript vont être discutés sous deux formes:
	\begin{itemize}
	\item{un apport scientifique, axé sur les résultats physiques de l'algorithme en lui-même dans le projet,}
	\item{ainsi qu'un apport plus axé sur la similitude entre cette nouvelle distribution, et celle créée lors du travail sur la première partie du stage}
	\end{itemize}

\subsubsection{Apport scientifique et biologique}

\subsubsection{Apport à la première partie du stage}

\chapter{Le travail en recherche}

La recherche m'intéressait au plus au point dès ma sortie du lycée.
\newline
Il était donc naturel que je puisses effectuer mon premier stage Universitaire dans une structure de recherche. Ayant effectué un DEUST de Biologie avant de postuler pour une licence d'Informatique, à l'Université Lille1, et étant toujours intéressé par ce domaine si particulier de la Science, j'étais vraiment très curieux de pouvoir effectuer mes premiers pas dans le monde de la recherche dans une équipe de bio-informaticiens, d'où mon postulat pour l'équipe Bonsai.

\subsection{Qualité de vie (relationnel et Scientifique)}

\subsubsection{La vie est un long fleuve tranquille...}

La vie en laboratoire de recherche a été un véritable plaisir.
\newline
L'intérêt de chaque membre de l'équipe envers le travail des autres et la facilité de communication vers tous a tout bonnement été exceptionnel - en effet, éternel curieux, je suis à la recherche perpétuelle d'informations de tous genres pouvant me permettre de progresser dans la vie, mais aussi de m'instruire/me cultiver. L'intérêt de chaque membre de l'équipe envers la Science a proprement parlé est imparable, chacun ayant une certaine spécifité qui en fait quelqu'un d'unique dans son domaine, lui permettant d'échanger avec tout le monde sur ses intérêts dans sa recherche de résultats, ou celles des autres.
\newline
Celà s'est ressenti de différentes manières:
\begin{itemize}
\item \textbf{les conférences/séminaires du Mardi matin}
	\newline
	Chaque mardi matin, généralement entre 11H et 12H, était réservé à une conférence/ un séminaire sur un sujet décidé par un membre de l'équipe Bonsai, ou par un intervenant extérieur.
	\newline
	Nous avons pu donc avoir le plaisir de recevoir des chercheurs de Pennsylvanie ou d'autres États de l'Amérique Centrale, venant nous parler de nouvelles méthodes de recherche, d'attrait à celle-ci, ou des dernières innovations et recherches effectuées/trouvées sur un sujet donné (\textit{the International Mouse Phenotyping Consortium}\footnote{Entreprise Internationale scientifique créant et caractérisant le phénotype de 20 000 \textit{knock-out} (ou "invalidation génétique" en Français) chez la souris}, les graphes de Bruijn\footnote{Graphe orienté qui permet de représenter les chevauchements de longueur (\textit{n}-1) entre tous les mots de longueur \textit{n} sur un alphabet donné - nom donné par le mathématicien les ayant décrit en 1946: Nicolaas Govert de Bruijn} ou encore la détection des variants complexes \footnote{Variants cytogénétiques constituant une majeure partie des cas de Leucémie myéloïde chronique}) -  mais pas seulement!
	\newline
	Cela a aussi permit à certains membres de l'équipe de pouvoir exposer ses travaux en cours ou finis, afin de pouvoir expliquer son cheminement, sa méthode de travail et ses résultats, comme par exemple le travail d'\textbf{Amandine Perrin} sur l' "Illustration dynamique des effets des vaccins et de la vaccination"\footnote{Travail réalisé entre Mars et Août 2013}.
\item \textbf{les entraides perpétuelles}
	\newline
	Chaque membre de l'équipe peut compter sur les autres s'il a un problème sur tel ou tel sujet - en effet, l'entraide est omniprésente dans l'équipe, que ce soit pour une aide informatique, algorithmique ou encore de compréhension biologique. Un bref exemple à donner serai celui m'ayant permis de rassembler plus de 5 algorithmes différents, de 5 personnes différentes, pour mon problème de représentation de graphes à 1000 arêtes.
\item \textbf{les débats sur les nouvelles données scientifiques}
	\newline
	Lors de la pause déjeuner, l'heure est à la discussion scientifique pour la plupart du temps, et les débats font rage, chacun exposant sa thèse, annonçant/dénonçant un ou plusieurs argument(s), etc...
\end{itemize}
Tout ceci m'a permis de me créer une belle idée de ce qu'est le monde de la recherche sur le plan relationnel et scientifique, et de me sentir beaucoup plus à l'aise dès la première semaine de stage.

\subsubsection{...ou pas}

Ce stage m'a permis aussi de me rendre compte de tous les problèmes tournant autour de la recherche, que ce soit pour l'accréditation des sujets de recherche, les problèmes liés aux actions en dehors de sa thèse, ou encore plus directement aux problèmes touchant à son sujet de thèse (la publication du sujet en est un exemple valable).
\newline
En effet, je me suis rendu compte qu'un travail de recherche représentait des heures incalculables à rechercher, programmer, analyser et rédiger - il faut en général entre 6 mois et 1 an pour arriver à des résultats satisfaisants, valant la peine de les publier.
\newline
La thèse mérite un temps considérablement long à être développé, approfondi et traité, afin de pouvoir enfin commencer réellement à travailler dessus - un peu plus longtemps qu'un nouveau sujet de recherche en général, avec l'expérience et les compétences/la culture en plus.
\newline
Tout ceci est lié à un rythme de publication assez rapide et drastique, qui ne permet pas réellement aux chercheurs de pouvoir travailler dans un certain confort psychologique.
\newline
La vigilance est aussi de mise quant à la publication d'autres scientifiques sur un sujet très proche de celui sur lequel un/plusieurs chercheur(s) travaille(nt). En effet, un article paraissant à une période très proche d'un autre traitant du même sujet, mais déjà publié, pourrai être refusé en publication pour ce motif - ce qui pourrai faire perdre au(x) chercheur(s) plusieurs mois (voire années) de travail à cause de celà.

\section{"Think different"}

\textcolor{red}{Ce qui me plaît, c'est de réfléchir et travailler autrement que les autres ("Think Different")!
\newline
Conditions de vie + relationnel OK (ne pas oublier de dire que les labos sont cools + matos cool
\newline
Conditions de travail Bof-Bof pour les chercheurs - OK pour moi...}

\chapter{Conclusion}


\chapter{Bibliographie}
\begin{itemize}
\item{Le premier article sur le projet \textit{Vidjil}, écrit par M. \textbf{Giraud Mathieu} et M. \textbf{Salson Mikaël}: \url{http://www.biomedcentral.com/1471-2164/15/409/abstract} \textit{(Consultation le 04 Avril 2014)}}
\item{\url{http://www.worldcat.org/title/recombinaison-vdj-illegitime-et-developpement-de-leucemies-aigues-lymphoblastiques-t/oclc/495056914} \textit{(Consultation le 10 Mai 2014)}}
\item{\url{http://fr.wikipedia.org/wiki/Recombinaison_V%28D%29J} \textit{(Consultation le 10 Mai 2014)}}
\item{\url{http://fr.wikipedia.org/wiki/DBSCAN} \textit{(Consultation le 19 Mai 2014)}}
\item{\url{http://citeseerx.ist.psu.edu/viewdoc/download?doi=10.1.1.88.4045&rep=rep1&type=pdf} \textit{(Consultation le 19 Mai 2014)}}
\item{\url{http://www.cise.ufl.edu/class/cis4930sp09dm/notes/dm5part4.pdf} \textit{(Consultation le 19 Mai 2014)}}
\end{itemize}

\end{document}